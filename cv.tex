\documentclass[12pt]{article}

% Remove page numbering
\pagestyle{empty}

% Allow hyperlinks
\usepackage{hyperref}

% Set hyperlink color and style
\hypersetup{colorlinks,breaklinks,urlcolor=blue,linkcolor=blue}

% Don't indent paragraphs
\setlength{\parindent}{0pt}

\begin{document}

%---------------------------------------
% Heading
%---------------------------------------

\begin{center}
\Huge{Michele Lindroos} \\
\normalsize{\today}
\end{center}

%---------------------------------------
% Personal Data
%---------------------------------------

\section*{Personal Data}
\begin{tabular}{rl}
Place and Date of Birth & Finland, June 25, 1987\\
Address & (undisclosed on github), Helsinki, Finland\\
Phone & (undisclosed on github)\\
E-mail & \href{mailto:michele.lindroos@gmail.com}{michele.lindroos@gmail.com}\\
GitHub & \href{https://github.com/keelefi}{keelefi}
\end{tabular}

%---------------------------------------
% Work Experience
%---------------------------------------

\section*{Work Experience}

\large{June 2012 - Current, LM Ericsson}\\
\normalsize{Software Developer.}\\

\large{Summer 2011, Aalto University, Acoustics Laboratory}\\
\normalsize{Research assistant at Research group of Paavo Alku (speech
processing).}\\

\large{Summer 2010, Aalto University, Acoustics Laboratory}\\
\normalsize{Research Assistant at Research group of Ville Pulkki (spatial
sound).}\\

\large{Autumn 2008 - Spring 2011, Aalto University, Laboratory of Mathematics}\\
\normalsize{Teacher of Basic Engineering Mathematics.}\\

\large{Spring 2008 - Autumn 2009, Oy Elmorex Ltd}\\
\normalsize{System Developer.}\\

%---------------------------------------
% Education
%---------------------------------------

\section*{Education}

\large{Master of Science (Tech.), Aalto University 2014}\\
\normalsize{
Major: Acoustics\\
Minor: Signal Processing\\
Study Programme: Telecommunications\\
Grade Point Average: 3.69\\
\\
Focused on studying computing hardware, real-time systems, programming 
languages and stochastic processes.\\
\\
Master's Thesis:
\href{https://aaltodoc.aalto.fi/bitstream/handle/123456789/14402/master_
Lindroos_Michele_2014.pdf?sequence=1}
{Task Scheduling in Multicore Conversational Speech Processing Systems}
}

%---------------------------------------
% Programming Languages
%---------------------------------------

\section*{Programming Languages}

\begin{tabular}{ll}
C & Excellent\\
C++11 and later & Very Strong\\
pre C++11 & Strong\\
Java & Strong\\
Python & Mediocre\\
(Object) Pascal & Strong\\
(Visual) Basic & Mediocre
\end{tabular}

%---------------------------------------
% Scripting Languages
%---------------------------------------

\section*{Scripting Languages}

\begin{tabular}{ll}
PHP & Strong\\
Tcl/Tk/Expect & Strong\\
JavaScript & Mediocre\\
Bash & Mediocre\\
Groovy & Basics\\
\end{tabular}

%---------------------------------------
% Assembly Languages
%---------------------------------------

\section*{Assembly Languages}

\begin{tabular}{ll}
MIPS & Mediocre\\
Freescale DSP & Mediocre\\
x86 (GAS) & Basics\\
ARM & Basics\\
\end{tabular}

%---------------------------------------
% Technologies and Expertise
%---------------------------------------

\section*{Technologies and Expertise}

\large{Linux Kernel}\\
\normalsize{Very familiar with the codebase. Especially strong knowledge of the
task scheduling architecture and algorithms.}\\

\large{Concurrency}\\
\normalsize{Very experienced in writing multithreaded software. Familiar with
synchronization primitives. Very knowledgeable in how to use and how
\emph{pthread} is implemented.}\\

\large{Real-time Software}\\
\normalsize{Familiar with the academic litterature. Experienced in considering,
measuring and tackling problems related to predictable response times.}\\

\large{Collaboration and Continuous Integration}\\
\normalsize{Experienced in using \emph{Git}, \emph{Gerrit}, \emph{CMake} and
\emph{Jenkins}.}\\

\large{Memory in Computer Science}\\
\normalsize{Familiar with memory abstractions, hardware memory designs, C++11
atomics, one glibc malloc/free implementation and Linux virtual memory
mapping.}\\

%---------------------------------------
% Languages
%---------------------------------------

\section*{Languages}
\begin{tabular}{ll}
Finnish & Excellent\\
Swedish & Excellent\\
English & Strong\\
Italian & Mediocre\\
\end{tabular}

\end{document}
